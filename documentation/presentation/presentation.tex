% @Author: Ramiro Luiz Nunes
% @Date:   2024-05-12 08:53:01
% @Last Modified by:   Ramiro Luiz Nunes
% @Last Modified time: 2024-05-12 08:53:08


% \documentclass{beamer} %voce pode usar este modelo tambem
\documentclass[handout,aspectratio = 169]{beamer}
\usefonttheme{serif}
\usepackage{graphicx,url}
\usepackage[brazil]{babel}   
\usepackage[utf8]{inputenc}
\usepackage[T1]{fontenc}
\usepackage{outlines}
\usepackage{comment}
\usepackage{algorithm}
\usepackage{algpseudocode}
\usepackage{subfig}
\usepackage{graphicx,url}
\usepackage{subfigure}


% \usepackage{fontspec}
% \setmainfont{Times}
%Image package

\usepackage{animate}
% \usepackage{hyphenat}
% \usepackage{multicol}
% \usepackage{pxfonts}

\batchmode
% \usepackage{pgfpages}
% \pgfpagesuselayout{4 on 1}[letterpaper,landscape,border shrink=5mm]
\usepackage{amsmath,amssymb,enumerate,epsfig,bbm,calc,color,ifthen,capt-of}
\usetheme{Berlin}
\usecolortheme{orchid}

\setbeamertemplate{caption}[numbered]
\setbeamertemplate{footline}[frame number]

%-------------------------Titulo/Autores/Orientador------------------------------------------------
\title{Árvores de Regressão: Processo de Treinamento}
\date{14 de maio de 2024}
\author[]{Introdução ao Aprendizado de Máquina \newline \newline Jean Marcelo Mira Junior \newline Ramiro Luiz Nunes}

%-------------------------Logo na parte de baixo do slide------------------------------------------
\pgfdeclareimage[height=1.0cm]{brasao}{brasao.pdf}
\logo{\pgfuseimage{brasao}\hspace*{0.5cm}}

\begin{document}
% -----------------------------------------------------------------------------

%---Gerador de Sumário---------------------------------------------------------
\frame{\titlepage}
\section[]{}
\begin{frame}{Sumário}
  \tableofcontents
\end{frame}
%---Fim do Sumário------------------------------------------------------------


% -----------------------------------------------------------------------------
\section{Introdução}

\begin{frame}{Desenvolvimento de um projeto de software embarcado}
    Para Kleppe et al. (2005), durante o desenvolvimento de software há:
    \begin{itemize}
        \item Levantamento de requisitos;
        \item Análise e descrição funcional;
        \item Elaboração do projeto;
        \item Codificação;
        \item Teste do código desenvolvido.
    \end{itemize}
\end{frame}

%------------------------------------------------------------------------------
\section{Objetivos}
\begin{frame}{Objetivos}
    \begin{outline}
        \1 Objetivo Geral:
        Elaborar um procedimento para geração de código utilizando modelos UML voltado para diagramas comportamentais, especificamente o diagrama de atividades.
    \end{outline}

    \begin{outline}
        \1 Objetivos Específicos:
        \2 Estudar a literatura sobre transformação de modelo para texto;
        \2 Selecionar uma abordagem para geração de código através de diagramas comportamentais UML;
        \2 Projetar e implementar a geração de código;
        \2 Testar e analisar os resultados da implementação.
    \end{outline}
\end{frame}

%------------------------------------------------------------------------------

\section{Fundamentação teórica}
\begin{frame}{Fundamentação teórica}
    \begin{itemize}
        \item Sistemas Embarcados;
        \item Desenvolvimento dirigido por modelos;
        \item Linguagem de modelagem unificada (UML);
        \item Geração de código.
    \end{itemize}
\end{frame}

\begin{frame}{UML e o diagrama comportamental de atividade - Nós}
    \begin{figure}
            \centering
             \includegraphics[width=0.5\linewidth]{Figs/nosdecontrole.png}
            \caption{Nós de controle (Autor, 2022).}
    \end{figure}
\end{frame}

\begin{frame}{UML e o diagrama comportamental de atividade - Ações}
    \begin{figure}
            \centering
             \includegraphics[width=0.7\linewidth]{Figs/atividadenos(2).png}
            \caption{Nós de atividade e ações (Autor, 2022).}
    \end{figure}
\end{frame}

%------------------------------------------------------------------------------
\section{Metodologia}
\begin{frame}{Conceito de trabalho}
    \begin{figure}
            \centering
             \includegraphics[width=0.7\linewidth]{Figs/metodo.png}
            \caption{Abordagem Aplicada para Geração de Código (Autor, 2022).}
        \end{figure}
\end{frame}

\begin{frame}{Exemplo diagrama de classe}
    \begin{figure}
            \centering
             \includegraphics[width=0.55\linewidth]{Figs/ProxyExample.png}
            \caption{Classe padrão proxy de hardware (Samek, 2008).}
        \end{figure}
\end{frame}

\begin{frame}{Modelagem no eclipse papyrus modeling}
    \begin{figure}
            \centering
             \includegraphics[width=0.75\linewidth]{Figs/classDiagramProxy.png}
            \caption{Diagrama de classe MotorProxy (Autor, 2022).}
        \end{figure}
\end{frame}

\begin{frame}{Modelagem do método accessMotorDirection - Parte I lógica}
    \begin{figure}
            \centering
             \includegraphics[width=0.45\linewidth]{Figs/accessMotorDirection.png}
            \caption{Diagrama de atividade accessMotorDirection (Autor, 2022).}
        \end{figure}
\end{frame}

\begin{frame}{Lógica para geração de código}
    \begin{figure}
            \centering
             \includegraphics[width=0.8\linewidth]{Figs/alg.png}
            \caption{Pseudocódigo para nó de decisão (Autor, 2022).}
        \end{figure}
\end{frame}

\begin{frame}{Modelagem do método accessMotorDirection - Parte II lógica}
    \begin{figure}
            \centering
             \includegraphics[width=0.45\linewidth]{Figs/accessMotorDirection.png}
            \caption{Diagrama de atividade accessMotorDirection (Autor, 2022).}
        \end{figure}
\end{frame}


%------------------------------------------------------------------------------

\section{Resultados}
\begin{frame}{DecisionNode e a relação com OpaqueBehavior}

\begin{figure}[htbp]
\centering
\subfigure[\centering Diagrama de atividade]{
\includegraphics[width=.4\textwidth]{Figs/accessMotorDirection.png}
} % end subfigure
\quad % dá um espaço entre as duas figuras.
\subfigure[\centering Código gerado]{
\includegraphics[width=.5\textwidth]{Figs/codCodaccessMotorDirection.png}
} % end subfigure
\caption{Exemplo 1 de código gerado accessMotorDirection (Autor, 2022).}
\end{figure}

\end{frame}

\begin{frame}{CreateObjectAction e CallOperationAction}

\begin{figure}[htbp]
\centering
\subfigure[\centering Diagrama de atividade]{
\includegraphics[width=.25\textwidth]{Figs/writeMotorSpeed.png}
} % end subfigure
\quad % dá um espaço entre as duas figuras.
\subfigure[\centering Código gerado]{
\includegraphics[width=.4\textwidth]{Figs/codwriteMotorSpeed.png}
} % end subfigure
\caption{Exemplo 2 de código gerado writeMotorSpeed (Autor, 2022).}
\end{figure}

\end{frame}

\begin{frame}{ReadStructuralFeatureAction e AddStructuralFeatureValueAction}

\begin{figure}[htbp]
\centering
\subfigure[\centering Diagrama de atividade]{
\includegraphics[width=.45\textwidth]{Figs/clearErrorStatus.png}
} % end subfigure
\quad % dá um espaço entre as duas figuras.
\subfigure[\centering Código gerado]{
\includegraphics[width=.45\textwidth]{Figs/codclearErrorStatus.png}
} % end subfigure
\caption{Exemplo 3 de código gerado clearErrorStatus (Autor, 2022).}
\end{figure}

\end{frame}

\begin{frame}{Exemplo diagrama de classe - Limitação}
    \begin{figure}
            \centering
             \includegraphics[width=0.52\linewidth]{Figs/buttonClass.png}
            \caption{Classe de padrão Debouncing (Samek, 2008).}
        \end{figure}
\end{frame}

\begin{frame}{Modelagem no eclipse papyrus modeling - Limitação}
    \begin{figure}
            \centering
             \includegraphics[width=0.8\linewidth]{Figs/classDiagramButton.png}
            \caption{Diagrama de classe Timer (Autor, 2022).}
        \end{figure}
\end{frame}

\begin{frame}{Modelagem diagrama de atividade delay - Limitação}
    \begin{figure}
            \centering
             \includegraphics[width=0.45\linewidth]{Figs/delay.png}
            \caption{Diagrama de atividade delay da classe Timer (Autor, 2022).}
        \end{figure}
\end{frame}

\begin{frame}{Comparação do código da literatura com código gerado}

\begin{figure}[htbp]
\centering
\subfigure[\centering Código da literatura (Samek, 2008).]{
\includegraphics[width=.45\textwidth]{Figs/timingCode.png}
} % end subfigure
\quad % dá um espaço entre as duas figuras.
\subfigure[\centering Código gerado (Autor, 2022).]{
\includegraphics[width=.45\textwidth]{Figs/meuTimer.png}
} % end subfigure
\caption{Exemplo 4 de código gerado método delay.}
\end{figure}

\end{frame}

%------------------------------------------------------------------------------

\section{Conclusões}

\begin{frame}{Conclusões}
    \begin{outline}
        \1 Após estudar a literatura de transformação de modelos para texto;
        \1 Selecionar a abordagem para geração de código embarcado através de diagrama comportamental de atividade UML; 
        \1 Realizar a implementação do gerador e obter fragmentos de código.
    \end{outline}
\end{frame}

\begin{frame}{Conclusões}
    \begin{outline}
        \1 A ferramenta consegue identificar e realizar a geração de código para os nós:
        \2 InitialNode;
        \2 DecisionNode;
        \2 MergeNode;
        \2 ActivityFinalNode.

        \1 Além das ações:
        \2 ReadStructuralFeatureAction;
        \2 AddStructuralFeatureValueAction;
        \2 ActivityParameterNode;
        \2 ClearStructuralFeatureAction;
        \2 CreateObjectAction;
        \2 DestroyObjectAction;
        \2 CallOperationAction.
    \end{outline}
\end{frame}

\begin{frame}{Conclusões}
    \begin{outline}
        \1 Conseguindo gerar fragmentos de código em C++ para:
        \2 Criar objetos;
        \2 Destruir objetos;
        \2 Atribuir valor aos objetos;
        \2 Retornar o valor dos objetos;
        \2 Estrutura If/Else;
        \2 Realizar a chamada de funções para os objetos.
    \end{outline}
\end{frame}

\begin{frame}{Conclusões}
    \begin{outline}
        \1 Tendo como limitações:
        \2 Tarefas paralelas (ForkNode e JoinNode);
        \2 Laço de repetição While e For (LoopNode ou DecisionNode);
        \2 OpaqueBehavior na condição do If;
        \2 Indentação.
    \end{outline}
\end{frame}

\begin{frame}{Conclusões}

    \begin{outline}
        \1 Trabalhos futuros:
        \2 Aumentar a quantidade de nós e ações comportamentais suportados (laços de repetição e tarefas em paralelo);
        \2 Controle do objeto enviado no fluxo de objetos;
        \2 Melhorar indentação.
    \end{outline}
\end{frame}


%------------------------------------------------------------------------------

\section{Referências}
\begin{frame}[t]{Referências}
    \begin{itemize}
        \item SAMEK, M. Practical UML Statecharts in C/C++: Event-Driven Programming for Embedded Systems. CRC Press, 2008. ISBN 9781482249262.
        \item KLEPPE, A.; WARMER, J.; BAST, W. MDA explained, the model driven architecture: practice and promise. 5. ed. Massachusetts: Pearson, 2005.
    \end{itemize}
\end{frame}

\begin{frame}{Fechamento}
    \centering
    Geração de Código Usando Diagramas de Atividade para Sistemas Embarcados
   \par\noindent\rule{\textwidth}{0.5pt}
    Jean Marcelo Mira Junior\\
    Orientador: Prof. Dr. Gian Ricardo Berkenbrock

\end{frame}

\end{document}

%-----------------------------------------------Este comentario nunca aparecera